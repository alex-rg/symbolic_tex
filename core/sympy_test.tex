\documentclass{article}

% Language setting
% Replace `english' with e.g. `spanish' to change the document language
\usepackage[english]{babel}
\usepackage{xfrac}

% Set page size and margins
% Replace `letterpaper' with `a4paper' for UK/EU standard size
\usepackage[letterpaper,top=2cm,bottom=2cm,left=3cm,right=3cm,marginparwidth=1.75cm]{geometry}

% Useful packages
\usepackage{amsmath}
\usepackage{graphicx}
\usepackage[colorlinks=true, allcolors=blue]{hyperref}

\title{Your Paper}
\author{You}

\begin{document}
\maketitle

\begin{abstract}
Your abstract.
\end{abstract}

\long\def\code#1{}
\long\def\codep#1{}


\section{Introduction}

$\codep{
f1 = TexFunction(name='f1', f_args=['x'], val='sin(x)')
f1(0)
}$

$\codep{
TexExpr("x + y*2")
}$

$\codep{
y1 = TexFraction('3/4')
y2 = TexFraction('1/2')
y1+y2
}$

Your introduction goes here! Simply start writing your document and use the Recompile button to view the updated PDF preview. Examples of commonly used commands and features are listed below, to help you get started.

$\codep{
x = TexSymbol('x').dot()
y = TexSymbol('y').hat()
expr = x**2 + y**2 
expr
}$

Once you're familiar with the editor, you can find various project setting in the Overleaf menu, accessed via the button in the very top left of the editor. To view tutorials, user guides, and further documentation, please visit our \href{https://www.overleaf.com/learn}{help library}, or head to our plans page to \href{https://www.overleaf.com/user/subscription/plans}{choose your plan}.

\section{Some examples to get started}
$\codep{
x = TexSymbol('x')
y = TexFraction('x/y')
z = TexNumber('1.5')
(x + y - z)*4
}$

\subsection{How to create Sections and Subsections}

Simply use the section and subsection commands, as in this example document! With Overleaf, all the formatting and numbering is handled automatically according to the template you've chosen. 


If you're using Rich Text mode, you can also create new section and subsections via the buttons in the editor toolbar.


\end{document}
